\documentclass[a4paper,twoside]{article}
\usepackage{blindtext}  
\usepackage{geometry}

% Chinese support
\usepackage[UTF8, scheme = plain]{ctex}

% Page margin layout
\geometry{left=2.3cm,right=2cm,top=2.5cm,bottom=2.0cm}


\usepackage{listings}
\usepackage{xcolor}
\usepackage{geometry}
\usepackage{amsmath}
\usepackage{float}
\usepackage{hyperref}

\usepackage{graphics}
\usepackage{graphicx}
\usepackage{subfigure}
\usepackage{epsfig}
\usepackage{float}

\usepackage{algorithm}
\usepackage[noend]{algpseudocode}

\usepackage{booktabs}
\usepackage{threeparttable}
\usepackage{longtable}
\usepackage{tikz}
\usepackage{multicol}

% cite package, to clean up citations in the main text. Do not remove.
\usepackage{cite}

\usepackage{color,xcolor}

%% The amssymb package provides various useful mathematical symbols
\usepackage{amssymb}
%% The amsthm package provides extended theorem environments
\usepackage{amsthm}
\usepackage{amsfonts}
\usepackage{enumerate}
\usepackage{enumitem}
\usepackage{listings}
\usepackage{minted}


\usepackage{indentfirst}
\setlength{\parindent}{2em} % Make two letter space in the first paragraph
\usepackage{setspace}
\linespread{1.5} % Line spacing setting
\usepackage{siunitx}
\setlength{\parskip}{0.5em} % Paragraph spacing setting

% \usepackage[contents =22920202204622, scale = 10, color = black, angle = 50, opacity = .10]{background}

\renewcommand{\figurename}{图}
\renewcommand{\listingscaption}{代码}
\renewcommand{\tablename}{表格}
\renewcommand{\contentsname}{目录}
\floatname{algorithm}{算法}

\graphicspath{ {images/} }

%%%%%%%%%%%%%
\newcommand{\StudentNumber}{22920202204622}  % Fill your student number here
\newcommand{\StudentName}{熊恪峥}  % Replace your name here
\newcommand{\PaperTitle}{实验(一)表达式翻译器}  % Change your paper title here
\newcommand{\PaperType}{编译原理} % Replace the type of your report here
\newcommand{\Date}{2023年3月17日}
\newcommand{\College}{信息学院}
\newcommand{\CourseName}{编译原理}
%%%%%%%%%%%%%

%% Page header and footer setting
\usepackage{fancyhdr}
\usepackage{lastpage}
\pagestyle{fancy}
\fancyhf{}
% This requires the document to be twoside
\fancyhead[LO]{\texttt{\StudentName }}
\fancyhead[LE]{\texttt{\StudentNumber}}
\fancyhead[C]{\texttt{\PaperTitle }}
\fancyhead[R]{\texttt{第{\thepage}页,共\pageref*{LastPage}页}}


\title{\PaperTitle}
\author{\StudentName}
\date{\Date}

\algnewcommand\algorithmicinput{\textbf{Input:}}
\algnewcommand\algorithmicoutput{\textbf{Output:}}
\algnewcommand\Input{\item[\algorithmicinput]}%
\algnewcommand\Output{\item[\algorithmicoutput]}%

\usetikzlibrary{positioning, shapes.geometric}

\begin{document}
	
%%%%%%%%%%%%%%%%%%%%%%%%%%%%%%%%%%%%%%%%%%%%
\makeatletter % change default title style
\renewcommand*\maketitle{%
	\begin{center} 
		\bfseries  % title 
		{\LARGE \@title \par}  % LARGE typesetting
		\vskip 1em  %  margin 1em
		{\global\let\author\@empty}  % no author information
		{\global\let\date\@empty}  % no date
		\thispagestyle{empty}   %  empty page style
	\end{center}%
	\setcounter{footnote}{0}%
}
\makeatother
%%%%%%%%%%%%%%%%%%%%%%%%%%%%%%%%%%%%%%%%%%%%
	
	
\thispagestyle{empty}

\vspace*{1cm}

\begin{figure}[htb]
	\centering
	\includegraphics[width=4.0cm]{logo.png}
\end{figure}

\vspace*{1cm}

\begin{center}
	\Huge{\textbf{\PaperType}}
	
	\Large{\PaperTitle}
\end{center}

\vspace*{1cm}

\begin{table}[H]
	\centering	
	\begin{Large}
		\renewcommand{\arraystretch}{1.5}
		\begin{tabular}{p{3cm} p{5cm}<{\centering}}
			姓\qquad 名 & \StudentName  \\
			\hline
			学\qquad号 & \StudentNumber \\
			\hline
			日\qquad期 & \Date  \\
			\hline
			学\qquad院 & \College  \\
			\hline
			课程名称 & \CourseName  \\
			\hline
		\end{tabular}
	\end{Large}
\end{table}

\newpage

\title{
	\Large{\textcolor{black}{\PaperTitle}}
}
	
	
\maketitle
	
\tableofcontents
 
\newpage
\setcounter{page}{1}

\begin{spacing}{1.2}

\section{选题}

在程序pInOrderToPost3.cpp的基础上,完成“二、 实验内容”中的要求。 

\section{实验目的}

构造一个中缀表达式到后缀表达式的翻译器,初步了解递归下降语法分析原理及语法制导翻译的过程。

\section{实现思路}

首先,为了支持变量、乘除法和括号,我们需要对文法进行修改,如公式\eqref{eqn:grammar}。其中,\texttt{print}函数用于输出后缀表达式,\texttt{identifer}表示变量名。
\begin{equation}
\label{eqn:grammar}
\begin{aligned}
expr \rightarrow & expr + term \ \{\mathop{print}('+')\} \\
 &| expr - term \ \{\mathop{print}('-')\} \\
 &| term \\
term \rightarrow & term * factor \ \{\mathop{print}('*')\} \\
 &| term / factor \ \{\mathop{print}('/')\} \\
 &| factor \\
factor \rightarrow & ( expr ) | item \\
item \rightarrow & 0  \ \{\mathop{print}('0')\} \\
 &| 1  \ \{\mathop{print}('1')\} \\
 &| \dots \\
 &| 9  \ \{\mathop{print}('9')\} \\
 &| identifer \ \{\mathop{print}(lexeme)\}
\end{aligned}
\end{equation}
其中,\texttt{identifer}的定义如\eqref{eqn:ident}。
\begin{equation}
\label{eqn:ident}
\begin{aligned}
identifer := & [a-zA-Z] [a-zA-Z0-9]*
\end{aligned}
\end{equation}
然后使用“语法制导定义”方法进行实现。

\section{问题1、如何使词法分析和语法分析的实现在形式上独立?}

为了在满足“按需词法分析”的要求下,使词法分析和语法分析的实现在形式上独立,我使用了C++的协程(coroutine)技术
\footnote[1]{编译使用C++协程的程序需要编译器支持C++20,需要使用VS2019以上的版本}。
协程是一种程序组织方式,它允许程序在执行过程中暂停,然后在稍后的某个时间点恢复执行。
节选的一段代码如代码~\ref*{code:coroutine}。当使用\texttt{co\_yield}时,程序会保留状态暂停执行,并返回指定的值。

程序返回的\texttt{generator<token>}对象可以被视为一个迭代器,可以使用\texttt{for}循环遍历。遍历时每当自增迭代器时,
程序会从上次暂停并返回的位置之后恢复执行,直到遇到下一个\texttt{co\_yield}语句暂停执行,
或者遇到\texttt{co\_return}结束执行。

因此,在调用\texttt{scan}函数时,虽然会立即返回,但是在遍历返回的\texttt{generator<token>}对象时真正的求值才会发生。
借助这一特性,可以在代码形式上使得词法分析和语法分析的实现独立。然而在执行逻辑上,词法分析仍是由语法分析驱动的。这样实现
能够在满足模块间“高内聚、低耦合”的要求下,依然保持了如同pInOrderToPost3.cpp中一样的执行逻辑。

\begin{listing}[htb]
	\caption{使用协程实现词法分析}
	\label{code:coroutine}
	\begin{minted}{c++}
std::experimental::generator<token> scan(std::string code)
{
	auto it = code.begin();
	while (it != code.end())
	{
		if (*it == ' ' || *it == '\t' || *it == '\n' || *it == '\r')
		{
			++it;
		}
		else if (std::isdigit(*it))
		{
			auto peek = it;
			while (std::isdigit(*peek))
			{
				++peek;
			}
			token t{};
			t.type = token_type::NUM;
			t.lexeme = std::string(it, peek);
			it = peek;
			co_yield t;
		}
		...
		else
		{
			throw std::runtime_error("Invalid token");
		}
	}
	co_return;
}
	\end{minted}
\end{listing}

\section{问题2、这样实现符合“语法制导翻译方案”吗?}

是的。

根据问题1的解释,当调用函数\texttt{scan}时,实际上的词法分析并没有发生。在语法分析的过程中,当需要一个token时,在调用
所返回的\texttt{generator}的迭代器的\texttt{operator++}函数时,才会发生一次求值,进而获取下一个\texttt{token}。
因此,虽然在代码的形式上类似于“进行完整的词法分析并返回了所有token”,
但实际上只是在语法分析的过程中进行了“按需词法分析”,得到一个token。符合
“语法制导翻译方案”。因为在执行逻辑上,词法分析仍是由语法分析驱动的。



\end{spacing}

\end{document}